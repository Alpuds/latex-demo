% xelatex
\documentclass[letterpaper, 12pt]{article}

%-----------------------------------------
\usepackage[utf8]{inputenc}
\usepackage[T1]{fontenc}
\usepackage{fontspec}
\usepackage{libertinus}
\usepackage[margin=1in]{geometry} % page layout
\usepackage{graphicx} % put images
\usepackage{amsmath} % Math support
\usepackage{tabularx} % more flexible tables
\usepackage{xeCJK} % Chinease, Japanese, and Korean encoding
\usepackage[pdfauthor={Alpuds},
            pdftitle={LaTeX Demonstration},
            pdfkeywords={LaTeX math Japanese}]{hyperref} % have links as well as pdf metadata

\setCJKmainfont{Noto Serif CJK JP}

\hypersetup{
    colorlinks=true,
    linkcolor=blue,
    filecolor=magenta,
    urlcolor=cyan
}

\newcommand{\q}[1]{``#1''} % macro for double quotes
%-----------------------------------------

\begin{document}
\title{{\LaTeX} Demonstration}
\author{Alpuds}
\maketitle
\tableofcontents
\newpage

\section{Learn \LaTeX}
To learn \LaTeX, go to \href{https://www.overleaf.com/learn}{Overleaf}.

\section{Math}
\subsection{Percent error}
\begin{equation}
    \left|\frac{\text{accepted value} - \text{lab}}{\text{accepted value}}\right| \times 100 = \%\:\text{error}
\end{equation}

This formula is used to determine the percent error.
With this, if the result is less than 10\%, then data recollection is not necessary.

\subsection{5\% rule}
\begin{equation}
    \left|\frac{\text{y-int}}{\text{y-max}}\right| \times 100
\end{equation}

This formula is used to determine if the y-intercept is negligible.
If the result is less than 5\%, then the y-intercept can be ignored.
This is because the y-intercept is insignificant.
If the result is greater than 5\%, then data recollection may be required.

\subsection{Exponent to radical}
\begin{equation}
    5^\frac{4}{3} = \sqrt[3]{5^4}
\end{equation}

The 3 is the index, the 5 is the base, and the 4 is the exponent.

\begin{equation}
    8^\frac{9}{2} = \sqrt{8^9}
\end{equation}

In this example, the 2 is omitted in the square root because the root symbol by itself is understood to mean square root.

\subsection{Functions}
\subsubsection{Linear}
\begin{equation}
    f(x) = mx + b
\end{equation}

A linear function means the line goes in a constant rate, thus linear.
$m$ is the rate, so the larger the number, the higher the rate of change is.
$b$ is the y-intercept. This is basically the starting number when $x = 0$.

\subsubsection{Exponential}
\begin{equation}
    f(x) = b^x
\end{equation}

Transformation: $g(x) = a(b)^{x-h} + k$

When the equation has $-h$, then the number is the amount that the exponential equation moved right.
A way to remember this is that it is the opposite of what you would think.
So, $-$ means left which means the opposite is right.
$+ k$ is the mount that the equation moved up.

\subsubsection{Quadratic}
\begin{equation}
    f(x) = ax^2 + bx + c
\end{equation}

\section{Text formatting}
This sentence has a \textbf{bold}, \textit{italic}, and \underline{underlined} text.

\section{Lists}
\subsection{Unordered list}
The following is a list of visual novel game companies:
\begin{itemize}
    \item Type-Moon
    \item Key
    \item ASa Project
    \item Nitro Plus
    \item Frontwing
\end{itemize}

\subsection{Ordered list}

The recommended route order in 月姫 (Tsukihime) is the following:
\begin{enumerate}
    \item アルクェイド (Arcueid)
    \item シエル (Ciel)
    \item 秋葉 (Akiha)
    \item 翡翠 (Hisui)
    \item 琥珀 (Kohaku)
\end{enumerate}

\subsection{Nested list}
\begin{itemize}
    \item{Key}
    \begin{enumerate}
        \item Clannad
        \item リトルバスターズ!(Little Busters!)
            \begin{itemize}
                \item クドわふたー (Kud Wafter)
            \end{itemize}
        \item リライト (Rewrite)
            \begin{itemize}
                \item リライト ハーヴェストフェスタ! (Rewrite Harvest festa!)
            \end{itemize}
    \end{enumerate}
    \item Type-Moon
    \begin{enumerate}
        \item Fate Stay/Night
            \begin{itemize}
                \item Fate/hollow ataraxia
            \end{itemize}
        \item 魔法使いの夜 (Mahoutsukai no Yoru; Which on the Holy Night)
        \item 月姫 (Tsukihime; Moon Princess)
            \begin{itemize}
                \item 歌月十夜 (Kagetsu Tooya)
            \end{itemize}
    \end{enumerate}
    \item Nitro Plus
    \begin{enumerate}
        \item Steins;Gate
        \item 沙耶の唄 (Saya no Uta; Saya's Song)
        \item 君と彼女と彼女の恋 (Kimi to Kanojo to Kanojo no koi; YOU and ME and HER)
    \end{enumerate}
\end{itemize}
\section{Japanese}
\begin{center}
        これは日本語です。\footnote{日本語 means "Japanese language"}
\end{center}
The above Japanese text means \q{This is Japanese}.
\end{document}
